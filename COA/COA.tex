\documentclass[12pt]{article}
\usepackage{hyperref}
\usepackage{amsmath, amssymb, amsfonts}
\usepackage[margin=1.5cm]{geometry}
\usepackage{xcolor}
\usepackage{graphicx}
\usepackage{xparse}
\usepackage{enumitem}
\parindent 0px
\newcommand{\lb}{\left|\rightarrow\right.}
\newcommand{\enter}{\\\textcolor{white}{1}}

\ExplSyntaxOn
\NewDocumentCommand{\bo}{m}
 {
   \bold_commas:n { #1 }
 }

\cs_new:Npn \bold_commas:n #1
 {
   \seq_set_split:Nnn \l_tmpa_seq { , } { #1 }
   \seq_map_indexed_function:NN \l_tmpa_seq \__bold_commas_aux:nn
 }

\cs_new:Npn \__bold_commas_aux:nn #1 #2
 {
   \textbf{#2}
   \int_compare:nNnTF { #1 } < { \seq_count:N \l_tmpa_seq }
     { , }
     { }
 }

\ExplSyntaxOff

\title{Computer Organization and Architecture}
\author{Me lol}
\date{\today}

\begin{document}
\maketitle
\pagebreak
\tableofcontents
\pagebreak

\section{Introduction}
\begin{center}(3 Hours/6 Marks)\end{center}
\subsection{Computer organization and architecture}
\begin{enumerate}[noitemsep, topsep = 0pt]
\item Define computer architecture.\hfill[2] (\bo{75 Ch}) [1.5] (81 Bh) [1] (\bo{72 Ch})
\item Define computer organization.\hfill[1.5] (\bo{\texttt{81 Bh}}) [1] (\bo{72 Ch})
\item Differentiate between computer organization and architecture.\hfill[2] (\bo{71 Ch}, 78 Ka) [3] (72 Ka)
\item Explain the design goals and performance metrics for a computer system regarding its organization and architecture.\hspace{9.9cm}[5] (76 Ash)
\end{enumerate}
\subsection{Structure and function}
\begin{enumerate}[noitemsep, topsep = 0pt]
\item Define structure and function of a computer system.\hfill[4] (80 Ba)\\
$\lb$Explain about the structural and functional viewpoint of a computer.\hfill[4] (\bo{\texttt{79 Bh}})
\item Explain the functional view and four types of operations used in computer. \hfill[6] (68 Ch)
\end{enumerate}
\subsection{Designing for performance}
\begin{enumerate}[noitemsep, topsep = 0pt]
\item What are the driving factors behind the need to design for performance?\hfill[4] (71 Shr)
\item How can we maintain a performance balance between processor and memory?\hfill[2] (\bo{72 Ch})\\
$\lb$ What is performance balance and why it is required?\hfill[3] (70 Asa)
\end{enumerate}
\subsection{Interconnection structures}
\begin{enumerate}[noitemsep, topsep = 0pt]
\item Explain the Interconnection structures of computer.\hfill[6] (75 Ash)\\ 
$\lb$Definition\hfill[2] (73 Shr)\\
$\lb$Explain different types.\hfill[4] (73 Shr)
\item Explain the interconnection of CPU with Memory and I/O devices along with different operations over them.\hfill[3+3] (\bo{70 Ch})
\end{enumerate}
\subsection{Bus interconnection}
\begin{enumerate}[noitemsep, topsep = 0pt]
\item What do you understand by Bus Interconnection.\hfill[2] (71 Shr)
\item What does the width of address bus represent in a system?\hfill[2] (\bo{75 Ch, 71 Ch})
\item Explain different elements of bus design.\hfill[2] (\bo{\texttt{79 Bh}}) [3] (70 Asa)
\item Discuss the limitations of using single bus system to connect different devices.
\enter\hfill[2] (\bo{75 Ch, 72 Ch})
\item Compare and explain the bus structure of typical computer system.\hfill[4] (78 Ka)
\item Explain different types of bus arbitration and compare them.\hfill[6] (\bo{\texttt{78 Bh}})
\item Why is bus hierarchy required?\hfill[2] (\bo{71 Ch})
\item Discuss about the usage of a Multiple Hierarchical Bus Architecture over single bus system.
\enter\hfill[6] (\bo{\texttt{80 Bh}})
\end{enumerate}
\subsection{PCI}
\begin{enumerate}[noitemsep, topsep = 0pt]
\item What is PCI?\hfill[1] (76 Ash)\\
$\lb$ Describe PCI bus configuration.\hfill[3] (81 Ba)
\end{enumerate}

\pagebreak
\section{Central Processing Unit}
\begin{center}(10 Hours/18 Marks)\end{center}
\subsection{CPU Structure and Function}
\begin{enumerate}[noitemsep, topsep = 0pt]
	\item Explain the component of CPU. \hfill [2] (\bo{\texttt{78 Ch}})
\item Draw the instruction cycle state diagram with example.\hfill[6] (\bo{76 Ch})\\
$\lb$Draw instruction cycle, state diagram with interrupt and explain it.\hfill[6] (\bo{74 Ch})\\
$\lb$Explain instruction cycle state diagram.\hfill[3] (81 Ba)\\
$\lb$Explain instruction cycle state diagram with interrupt handling.\hfill[2] (80 Ba) [3] (\bo{\texttt{81 Bh}})\\
$\lb$Explain the computer functions with different cycles.\hfill[3] (72 Ka)

	\item Explain the general organization of register in CPU. \hfill [6] (\bo{71 Ch})
\end{enumerate}
\subsection{Arithmetic and Logic Unit}
\begin{enumerate}[noitemsep, topsep = 0pt]
	\item Design a 1-bit ALU which can perform addition, AND, OR, and X-OR operations. Explain the different types of instruction formats. \hfill [4+4] (\bo{\texttt{80 Bh}})
	
	\item Design a 2-bit ALU that can perform subtraction, AND, OR and XOR. \hfill [8] (\bo{75 Ch})
	
	\item What are the stages of ALU design? Explain with the example of 2-bit ALU performing addition, subtraction, OR and XOR. \hfill [8] (70 Asa)
\end{enumerate}
\subsection{Instruction Formats}
\begin{enumerate}[noitemsep, topsep = 0pt]
	\item What do you mean by instruction format? \hfill [4] (\bo{72 Ch})
	
	\item What are the different types of instructions? \hfill [3] (\bo{71 Ch})
	
	\item Explain Instruction Format with its types? \hfill [2] (71 Shr)
	
	\item What are the most common fields in an instruction. \hfill [2] (\bo{68 Ch})
	
	\item Write down the code to evaluate in three address, two address, one address and zero address instruction format.
	\begin{enumerate}[noitemsep, topsep = 0pt, label = \alph*.]
		\item Y = (A-B/C)*[D+(E*G)] \hfill [8] (\bo{\texttt{81 Bh}, 76 Ch})
		\item N = ((P-QxR)/S)+(T/U)+VxW) \hfill [8] (81 Ba)
		\item $\text{X} = \dfrac{\text{A-B+Cx(DxE-F})}{\text{G+HxK}}$ \hfill [8] (80 Ba)
		\item X = ((A+B)/C) + (D-E) \hfill [8] (\bo{\texttt{79 Bh}})
		\item Y = (A-B/C) x (D+ExG)/F \hfill [8] (78 Ka)
		\item Y = (W+X) * (Y-Z) \hfill [8] (76 Ash)
		\item Y = (A+B)/C + D/(E*F) \hfill [8] (\bo{75 Ch})
		\item (\textit{In present sources, operation is not given. If found, please contact})\hfill [8] (75 Ash)
		\item X = (P+Q) x (R+S) \hfill [8] (\bo{74 Ch})
		\item Y = A * (B+D/C)+(G*E)/F \hfill [8] (73 Shr)
		\item X = (A-B*F)*C+D/E \hfill [8] (\bo{72 Ch})
		\item Y = (A+B)*(C+D)+G/E*F \hfill [8] (72 Ka)
		\item X = (A+B) x (C+D) \hfill [5] (\bo{71 Ch}) [6] (\bo{68 Ch}, 71 Shr)
		\item Y = A/B+(CxD) + F(H/G) \hfill [8] (\bo{70 Ch}
		\item Y = A+(B*C)+D \hfill [8] (\bo{68 Ba})
		\item Y = AB + (F/G) + CD [8] (\bo{67 Asa})
		
	\end{enumerate}
\end{enumerate}
\subsection{Data Transfer and Manipulation}
\begin{enumerate}[noitemsep, topsep = 0pt]
	\item Explain data transfer instruction with example. \hfill [4] (\bo{\texttt{81 Bh}})
	
	\item Explain different types of data manipulation instructions with example. \hfill [8] (\bo{\texttt{78 Bh}})\\
	$\lb$What are the three types of data manipulation instructions used in computer? Explain. 
	\enter\hfill [8] (\bo{67 Asa})
\end{enumerate}
\subsection{Addressing Modes}
\begin{enumerate}[noitemsep, topsep = 0pt]
	\item What is addressing mode? \hfill [2] (\bo{\texttt{80 Bh}, 76 Ch, 68 Ch})
	
	\item Differentiate between Immediate and direct addressing modes. \hfill [4] (\bo{\texttt{81 Bh}})
	
	\item Write down the need for addressing modes. \hfill [2] (\bo{74 Ch})
	
	\item Comparision of different types of addressing modes. \hfill [6] (\bo{76 Ch}) [8] (76 Ash) [10] (72 Ka)\\
	$\lb$with adv/disadv. \hfill [10] (78 Ka)\\
	$\lb$with algorithm as well as adv/disadv. \hfill [8] (\bo{68 Ba})
	
	\item Write down different types of addressing mode and:\\
	$\lb$Explain with adv/disadv. \hfill [8] (81 Ba, 80 Ba) [10] (\bo{70 Ch})\\
	$\lb$Explain with suitable example. \hfill [6] (\bo{\texttt{80 Bh, 79 Bh}, 74 Ch}) [8] (\bo{68 Ch}, 70 Asa)
		
	\item Following instructions are give: \hfill [10] (73 Shr)
	\begin{enumerate}[noitemsep, topsep = 0pt, label = \alph*.]
		\item LDA 2000H
		\item MVI B, 32H
		\item STAX D
		\item MOV A, B
	\end{enumerate}
	Which addressing modes are used in the above instructions? Explain briefly about them.
	
	\item Describe the operation of LD (load) instruction under various addressing modes with syntax. 
	\enter\hfill [4] (\bo{71 Ch})
\end{enumerate}
\subsection{RISC and CISC}
\begin{enumerate}[noitemsep, topsep = 0pt]
	\item Comparison between RISC and CISC architecture. \hfill [6] (\bo{\texttt{78 Bh}, 72 Ch}, 75 Ash)
\end{enumerate}
\subsection{64 – Bit Processor}

\pagebreak
\section{Control Unit}
\begin{center}(6 Hours/10 Marks)\end{center} 
\subsection{Control Memory}
\subsection{Addressing sequencing}
\subsection{Computer configuration}
\subsection{Microinstruction Format}
\subsection{Symbolic Microinstructions}
\subsection{Symbolic Microprogram}
\subsection{Control Unit Operation}
\subsection{Design of Control Unit}

\pagebreak
\section{Pipeline and Vector processing}
\begin{center}(5 Hours/10 Marks)\end{center}
\subsection{Pipelining}
\subsection{Parallel Processing}
\subsection{Arithmetic Pipeline}
\subsection{Instruction Pipeline}
\subsection{RISC Pipeline}
\subsection{Vector Processing}
\subsection{Array Processing}

\pagebreak
\section{Computer Arithmetic}
\begin{center}(8 Hours/14 Marks)\end{center}
\subsection{Addition Algorithm}
\subsection{Subtraction Algorithm}
\subsection{Multiplication Algorithm}
\begin{enumerate}[noitemsep, topsep=0pt]
	\item Draw a flowchart for Booth's multiplication algorithm for signed multiplication. \hfill [5] (\bo{\texttt{81 Bh}})
\end{enumerate}
\subsection{Division Algorithm}
\subsection{Logical Operation}

\pagebreak
\section{Memory System}
\begin{center}(5 Hours/8 Marks)\end{center}
\subsection{Microcomputer Memory}
\subsection{Characteristics of memory systems}
\subsection{The Memory Hierarchy}
\subsection{Internal and External memory}
\subsection{Cache memory principles}
\subsection{Elements of Cache design}
\subsubsection{Cache size}
\subsubsection{Mapping function}
\subsubsection{Replacement algorithm}
\subsubsection{Write policy}
\subsubsection{Number of caches}

\pagebreak
\section{Input-Output organization}
\begin{center}(6 Hours/10 Marks)\end{center}
\subsection{Peripheral devices}
\subsection{I/O modules}
\subsection{Input-Output interface}
\subsection{Modes of transfer}
\subsubsection{Programmed I/O}
\subsubsection{Interrupt-driven I/O}
\subsubsection{Direct Memory access}
\subsection{I/O Processors}
\subsection{Data Communication Processor}

\pagebreak
\section{Multiprocessors}
\begin{center}(2 Hours/4 Marks)\end{center}
\subsection{Characteristics of multiprocessors}
\subsection{Interconnection Structures}
\subsection{Interprocessor Communication and Synchronization}
\end{document}
