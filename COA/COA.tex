\documentclass[12pt]{article}
\usepackage{hyperref}
\usepackage{amsmath, amssymb, amsfonts}
\usepackage[margin=1.5cm]{geometry}
\usepackage{xcolor}
\usepackage{graphicx}
\usepackage{xparse}
\usepackage{enumitem}
\usepackage{inconsolata}
\parindent 0px
\newcommand{\lb}{\\$\left|\rightarrow\right.$}
\newcommand{\enter}{\\\textcolor{white}{1}}

\ExplSyntaxOn
\NewDocumentCommand{\bo}{m}
 {
   \bold_commas:n { #1 }
 }

\cs_new:Npn \bold_commas:n #1
 {
   \seq_set_split:Nnn \l_tmpa_seq { , } { #1 }
   \seq_map_indexed_function:NN \l_tmpa_seq \__bold_commas_aux:nn
 }

\cs_new:Npn \__bold_commas_aux:nn #1 #2
 {
   \textbf{#2}
   \int_compare:nNnTF { #1 } < { \seq_count:N \l_tmpa_seq }
     { , }
     { }
 }

\ExplSyntaxOff

\title{Computer Organization and Architecture}
\author{Me lol}
\date{\today}

\begin{document}
\maketitle

\vspace{13cm}
\begin{large}\textbf{Notes}\end{large}
\begin{itemize}
\item PYQs of BEX/BEI/BCT's CT603 are combined.
\item BEX's and BCT's are kept with normal font.\item BEI's are kept with \texttt{this styling to differentiate}.
\item Regular exam's questions are kept as \bo{bold} while back exam are kept as normal font.
\item Months are marked as: 
\begin{itemize}[noitemsep]
	\item Ba: Baisakh
	\item Jth: Jestha
	\item Asa: Ashar
	\item Shr: Shrawan
	\item Bh: Bhadra
	\item Ash: Ashwin
	\item Ka: Kartik
	\item Mng: Mangsir
	\item Po: Poush
	\item Ma: Magh
	\item Ch: Chaitra
\end{itemize}
\end{itemize}
\pagebreak

\tableofcontents
\pagebreak

\section{Introduction}
	\begin{center}(3 Hours/6 Marks)\end{center}
	\subsection{Computer organization and architecture}
	\begin{enumerate}[noitemsep, topsep = 0pt]
	\item Define computer architecture.\hfill[2] (\bo{75 Ch}) [1.5] (81 Bh) [1] (\bo{72 Ch})
	\item Define computer organization.\hfill[1.5] (\bo{\texttt{81 Bh}}) [1] (\bo{72 Ch})
	\item Differentiate between computer organization and architecture.\hfill[2] (\bo{71 Ch}, 78 Ka) [3] (72 Ka)
	\item Explain the design goals and performance metrics for a computer system regarding its organization and architecture.\hspace{9.9cm}[5] (76 Ash)
	\end{enumerate}
	\subsection{Structure and function}
	\begin{enumerate}[noitemsep, topsep = 0pt]
	\item Define structure and function of a computer system.\hfill[4] (80 Ba)
	\lb Explain about the structural and functional viewpoint of a computer.\hfill[4] (\bo{\texttt{79 Bh}})
	\item Explain the functional view and four types of operations used in computer. \hfill[6] (68 Ch)
	\end{enumerate}
	\subsection{Designing for performance}
	\begin{enumerate}[noitemsep, topsep = 0pt]
	\item What are the driving factors behind the need to design for performance?\hfill[4] (71 Shr)
	\item How can we maintain a performance balance between processor and memory?\hfill[2] (\bo{72 Ch})
	\lb What is performance balance and why it is required?\hfill[3] (70 Asa)
	\end{enumerate}
	\subsection{Interconnection structures}
	\begin{enumerate}[noitemsep, topsep = 0pt]
	\item Explain the Interconnection structures of computer.\hfill[6] (75 Ash)\
	\lb Definition\hfill[2] (73 Shr)
	\lb Explain different types.\hfill[4] (73 Shr)
	\item Explain the interconnection of CPU with Memory and I/O devices along with different operations over them.\hfill[3+3] (\bo{70 Ch})
	\end{enumerate}
	\subsection{Bus interconnection}
	\begin{enumerate}[noitemsep, topsep = 0pt]
	\item What do you understand by Bus Interconnection.\hfill[2] (71 Shr)
	\item What does the width of address bus represent in a system?\hfill[2] (\bo{75 Ch, 71 Ch})
	\item Explain different elements of bus design.\hfill[2] (\bo{\texttt{79 Bh}}) [3] (70 Asa)
	\item Discuss the limitations of using single bus system to connect different devices.
	\enter\hfill[2] (\bo{75 Ch, 72 Ch})
	\item Compare and explain the bus structure of typical computer system.\hfill[4] (78 Ka)
	\item Explain different types of bus arbitration and compare them.\hfill[6] (\bo{\texttt{78 Bh}})
	\item Why is bus hierarchy required?\hfill[2] (\bo{71 Ch})
	\item Discuss about the usage of a Multiple Hierarchical Bus Architecture over single bus system.
	\enter\hfill[6] (\bo{\texttt{80 Bh}})
	\end{enumerate}
	\subsection{PCI}
	\begin{enumerate}[noitemsep, topsep = 0pt]
	\item What is PCI?\hfill[1] (76 Ash)
	\lb Describe PCI bus configuration.\hfill[3] (81 Ba)
	\end{enumerate}

	\pagebreak
\section{Central Processing Unit}
	\begin{center}(10 Hours/18 Marks)\end{center}
	\subsection{CPU Structure and Function}
	\begin{enumerate}
		\item Explain the component of CPU. \hfill [2] (\bo{\texttt{78 Ch}})
	\item Draw the instruction cycle state diagram with example.\hfill[6] (\bo{76 Ch})
	\lb Draw instruction cycle, state diagram with interrupt and explain it.\hfill[6] (\bo{74 Ch})
	\lb Explain instruction cycle state diagram.\hfill[3] (81 Ba)
	\lb Explain instruction cycle state diagram with interrupt handling.\hfill[2] (80 Ba) [3] (\bo{\texttt{81 Bh}})
	\lb Explain the computer functions with different cycles.\hfill[3] (72 Ka)

		\item Explain the general organization of register in CPU. \hfill [6] (\bo{71 Ch})
	\end{enumerate}
	
	\subsection{Arithmetic and Logic Unit}
	\begin{enumerate}
		\item What are the stages of ALU design? \hfill [2] (70 Asa)
	
		\item Design a 1-bit ALU which can perform addition, AND, OR, and X-OR operations. \hfill [4] (\bo{\texttt{80 Bh}})
		
		\item Design a 2-bit ALU that can perform subtraction, AND, OR and XOR. \hfill [8] (\bo{75 Ch})
		\lb 2-bit ALU performing addition, subtraction, OR and XOR. \hfill [6] (70 Asa)
	\end{enumerate}
	\subsection{Instruction Formats}
	\begin{enumerate}
		\item What do you mean by instruction format? \hfill [4] (\bo{72 Ch})
		
		\item Explain the different types of instruction formats. \hfill [3] (\bo{71 Ch}) [4] (\bo{\texttt{80 Bh}})
		
		\item Explain Instruction Format with its types? \hfill [2] (71 Shr)
		
		\item What are the most common fields in an instruction. \hfill [2] (\bo{68 Ch})
		
		\item Write down the code to evaluate in three address, two address, one address and zero address instruction format.
		\begin{enumerate}[noitemsep, topsep = 0pt, label = \alph*.]
			\item Y = (A-B/C)*[D+(E*G)] \hfill [8] (\bo{\texttt{81 Bh}, 76 Ch})
			\item X = ((A+B)/C) + (D-E) \hfill [8] (\bo{\texttt{79 Bh}})
			\item Y = (A+B)/C + D/(E*F) \hfill [8] (\bo{75 Ch})
			\item X = (P+Q) x (R+S) \hfill [8] (\bo{74 Ch})
			\item X = (A-B*F)*C+D/E \hfill [8] (\bo{72 Ch})
			\item X = (A+B) x (C+D) \hfill [5] (\bo{71 Ch}) [6] (\bo{68 Ch}, 71 Shr)
			\item Y = A/B+(CxD) + F(H/G) \hfill [8] (\bo{70 Ch}
			\item Y = A+(B*C)+D \hfill [8] (\bo{68 Ba})
			\item Y = AB + (F/G) + CD \hfill [8] (\bo{67 Asa})
			\item N = ((P-QxR)/S)+(T/U)+VxW) \hfill [8] (81 Ba)
			\item $\text{X} = \dfrac{\text{A-B+Cx(DxE-F})}{\text{G+HxK}}$ \hfill [8] (80 Ba)
			\item Y = (A-B/C) x (D+ExG)/F \hfill [8] (78 Ka)
			\item Y = (W+X) * (Y-Z) \hfill [8] (76 Ash)
			\item (\textit{In present sources, operation is not given. If found, please contact})\hfill [8] (75 Ash)
			\item Y = A * (B+D/C)+(G*E)/F \hfill [8] (73 Shr)
			\item Y = (A+B)*(C+D)+G/E*F \hfill [8] (72 Ka)
			
		\end{enumerate}
	\end{enumerate}
	\subsection{Data Transfer and Manipulation}
	\begin{enumerate}
		\item Explain data transfer instruction with example. \hfill [4] (\bo{\texttt{81 Bh}})
		
		\item Explain different types of data manipulation instructions with example. \hfill [8] (\bo{\texttt{78 Bh}})
		\lb What are the three types of data manipulation instructions used in computer? Explain. 
		\enter\hfill [8] (\bo{67 Asa})
	\end{enumerate}
	\subsection{Addressing Modes}
	\begin{enumerate}
		\item What is addressing mode? \hfill [2] (\bo{\texttt{80 Bh}, 76 Ch, 68 Ch})
		
		\item Differentiate between Immediate and direct addressing modes. \hfill [4] (\bo{\texttt{81 Bh}})
		
		\item Write down the need for addressing modes. \hfill [2] (\bo{74 Ch})
		
		\item Comparision of different types of addressing modes. \hfill [6] (\bo{76 Ch}) [8] (76 Ash) [10] (72 Ka)
		\lb with adv/disadv. \hfill [10] (78 Ka)
		\lb with algorithm as well as adv/disadv. \hfill [8] (\bo{68 Ba})
		
		\item Write down different types of addressing mode and:
		\lb Explain with adv/disadv. \hfill [8] (81 Ba, 80 Ba) [10] (\bo{70 Ch})
		\lb Explain with suitable example. \hfill [6] (\bo{\texttt{80 Bh, 79 Bh}, 74 Ch}) [8] (\bo{68 Ch}, 70 Asa)
			
		\item Following instructions are give: \hfill [10] (73 Shr)
		\begin{enumerate}[noitemsep, topsep = 0pt, label = \alph*.]
			\item LDA 2000H
			\item MVI B, 32H
			\item STAX D
			\item MOV A, B
		\end{enumerate}
		Which addressing modes are used in the above instructions? Explain briefly about them.
		
		\item Describe the operation of LD (load) instruction under various addressing modes with syntax. 
		\enter\hfill [4] (\bo{71 Ch})
	\end{enumerate}
	\subsection{RISC and CISC}
	\begin{enumerate}
		\item Comparison between RISC and CISC architecture. \hfill [4] (71 Shr) [6] (\bo{\texttt{78 Bh}, 72 Ch}, 75 Ash)
	\end{enumerate}
	\subsection{64 – Bit Processor}

	\pagebreak
\section{Control Unit}
	\begin{center}(6 Hours/10 Marks)\end{center} 
	\subsection{Control Memory}
	
	\subsection{Addressing sequencing}
		\begin{enumerate}
			\item Explain address sequencing with the help of a block diagram. \hfill [5] (\bo{\texttt{80 Bh}})
			
			\item What is address sequencing? \hfill [3] (\bo{71 Ch, 67 Asa})
			
			\item How does a sequencing logic work in micro-programmed control unit to execute a micro-program?
			\enter\hfill [6] (\bo{70 Ch})
			\item Explain the address sequencing capabilities required in a control memory. \hfill [5] (\bo{67 Asa})
		\end{enumerate}
	
	\subsection{Computer configuration}
	
	\subsection{Microinstruction Format}
		\begin{enumerate}[noitemsep, topsep=0pt]
			\item Explain the microinstruction format.  \hfill [3] (\bo{\texttt{80 Bh}}) [4] (\texttt{81 Ba}) [5] (72 Ka)
			\lb with example. \hfill [5] (\bo{\texttt{81 Bh}}) [6] (71 Shr)
			\lb Explain various fields in micro-instruction format with neat and clean block diagram.
			\enter\hfill [3] (\bo{68 Ch})
		\end{enumerate}
		
	\subsection{Symbolic Microinstructions}
		\begin{enumerate}[noitemsep, topsep=0pt]
			\item How address of micro instruction is generated by next address generator in control unit? Explain with suitable diagram. \hfill [8] (\bo{76 Ch})
		\end{enumerate}
	
	\subsection{Symbolic Microprogram}
		\begin{enumerate}[noitemsep, topsep=0pt]
			\item Write a microprogram for the fetch cycle and addition cycle. \hfill [5] (\bo{\texttt{81 Bh}})
			
			\item Differentiate between symbolic and bianry micro instruction. \hfill [4] (\texttt{81 Ba})
			
			\item Explain the operation of microprogram sequencer used in microprogrammed control unit. 
			\enter\hfill [5] (\bo{\texttt{79 Bh}})
			
			\item Explain with diagram the working of microprogram sequencer for control memory. \hfill [6] (78 Ka)
			
			\item Write down the symbolic microprogram for fetch routine and addition execute routine.
			\enter\hfill [4] (78 Ka)
			
			\item Describe various fields in micro-instruction format with diagram showing different fields.
			\enter\hfill [6] (76 Ash)
			
			\item Write micro program for fetch cycle. \hfill [4] (73 Shr)
		\end{enumerate}
		
	\subsection{Control Unit Operation}
		\begin{enumerate}[noitemsep, topsep=0pt]
			\item What are the types of control signals? \hfill [3] (\bo{68 Ba})
		\end{enumerate}
	
	\subsection{Design of Control Unit}
		\begin{enumerate}[noitemsep, topsep=0pt]
			\item Differentiate between hardwired and microprogrammed control unit. 
			\enter\hfill [4] (\bo{74 Ch}, 70 Asa) [5] (\bo{75 Ch, 70 Ch}, \texttt{80 Ba})
			
			\item Describe the operation of hardwired control unit with a typical diagram. \hfill [5] (\bo{\texttt{79 Bh}})
			\lb Explain the key steps of hardware implementation of control unit. \hfill [7] (\bo{68 Ba})
			
			\item Explain microprogrammed control unit with block diagram. \hfill [5] (\texttt{80 Ba})
			
			\item Explain the organization structure of a microprogram control unit and the generation of control signals using microprogram. \hfill [10] (\bo{\texttt{78 Bh}})
			
			\item Explain block diagram of micro-programmed control organization. \hfill [4] (76 Ash)
			
			\item Draw and explain block diagram of micro-programmed sequencer for control memory.
			\enter\hfill [5] (\bo{75 Ch})	
			\lb Draw the diagram of Micro-programmed sequencer for a control memory and explain it. 
			\enter\hfill [10] (75 Ash)
			\lb Explain the micro program sequencer used in microprogrammed control unit. \hfill [6] (\bo{74 Ch})
			
			\item Explain microinstruction format used in microprogramming control unit. \hfill [6] (73 Shr)
			
			\item What factors cause micro-programmed control unit to be selected over hardwired control unit.
			\enter\hfill [3] (\bo{72 Ch})
			
			\item Explain with block diagram, how address of control memory is selected in micro-programmed control unit. \hfill [7] (\bo{72 Ch})
			\lb Describe how address of control memory is selected. \hfill [7] (\bo{68 Ch})
			
			\item Explain the address sequencer with the help of a block diagram. \hfill [5] (72 Ka)
			
			\item Explain the selection of address for control memory with its block diagram. \hfill [7] (\bo{71 Ch})
			
			\item Explain the organization of a control memory. \hfill [4] (71 Shr)
			
			\item Explain with steps involved when you are designing micro-program control unit. \hfill [6] (70 Asa)
		\end{enumerate}

	\pagebreak

\section{Pipeline and Vector processing}
	\begin{center}(5 Hours/10 Marks)\end{center}
	\subsection{Pipelining}
		\begin{enumerate}
			\item What is Pipelining? \hfill [1] (\bo{\texttt{81 Bh}})
			\lb Define pipeline. \hfill [1] (72 Ka)

			\item Explain types of pipelining. \hfill [3] (72 Ka)

			\item Explain about the different types of conflicts that can be seen in a pipeline. \hfill [6] (\bo{\texttt{80 Bh}})

			\item How can we prove that pipelining improves the performance of a computer? \hfill [4] (\texttt{81 Ba})

			\item Derive the expression showing speed up ratio equals number of segments in pipeline. \hfill [3] (\bo{75 Ch})

			\item What is meant by hazard in pipelining? \hfill [2] (76 Ash) [4] (\bo{\texttt{78 Bh}})

			\item Describe different types of pipeline hazards. \hfill [4] (76 Ash)
			\lb with example. \hfill [6] (\bo{\texttt{79 Bh}}, 72 Ka)

			\item How can you overcome hazards? \hfill [2] (76 Ash)

			\item Explain with example data and control hazards in pipeline conflict. \hfill [6] (\bo{\texttt{78 Bh}})

			\item Explain control pipeline hazard and its solutions. \hfill [6] (\bo{72 Ch})

			\item What is instruction hazard in pipeline? \hfill [2] (70 Asa)

			\item Discuss in detail about data dependency problem that arises in pipelining along with its solution.
			\enter\hfill [5] (\bo{75 Ch})
		\end{enumerate}

	\subsection{Parallel Processing}
		\begin{enumerate}
			\item Explain the Flynn's classification of computer system. \hfill [4] (\bo{\texttt{81 Bh}, 72 Ch})

			\item Discuss about parallel processing. \hfill [4] (71 Shr)

			\item How parallel processing can be achieved in pipelining, explain it with time-space diagram for four segments pipeline having six tasks. \hfill [6] (71 Shr)
		\end{enumerate}

	\subsection{Arithmetic Pipeline}
		\begin{enumerate}
			\item Explain the instruction pipeline with example. \hfill [4] (\bo{\texttt{79 Bh}}) [5] (\bo{71 Ch, 70 Ch})
			\lb Explain in detail how the arithmetic pipeline increases the performance of a system.
			\enter\hfill [7] (73 Shr)

			\item Explain arithmetic pipeline for solving floating-point addition/subtraction. \hfill [5] (\texttt{81 Ba})
			
		\end{enumerate}

	\subsection{Instruction Pipeline}
		\begin{enumerate}
			\item Explain the instruction pipeline with example. \hfill [5] (\bo{71 Ch, 70 Ch})

			\item Describe four stage instruction pipeline. \hfill [4] (\bo{\texttt{81 Bh}}) [5] (\bo{76 Ch}, 78 Ka)
			\lb Explain with example. \hfill [8] (70 Asa)
			
			\item How pipeline processing is done in an instruction pipeline? \hfill [3] (78 Ka)

			\item Construct time-space diagram for four instructions with four-stage pipeline and show how pipelining reduces the execution time? \hfill [5] (\texttt{81 Ba})
			\lb Explain the operation of instruction pipeline for processing four segment instruction cycle with the help of space-time diagram. \hfill [6] (\texttt{81 Ba})

			\item Show that the speedup factor for a pipelined processor is equal to the number of stages in a pipeline. \hfill [4] (\bo{\texttt{80 Bh}})

			\item Draw a time-space diagram for four segments having six tasks. \hfill [6] (\bo{76 Ch})

			\item Explain six stage instruction pipeline with example. \hfill [10] (75 Ash)
		\end{enumerate}

	\subsection{RISC Pipeline}
		\begin{enumerate}
			\item RISC has the ability to use efficient instruction pipeline. Justify. \hfill [3] (73 Shr)
		\end{enumerate}

	\subsection{Vector Processing}
	\subsection{Array Processing}

	\pagebreak

\section{Computer Arithmetic}
	\begin{center}(8 Hours/14 Marks)\end{center}
	\subsection{Addition Algorithm}
	\begin{enumerate}[noitemsep, topsep=0pt]
		\item Explain the floating-point addition and subtraction process
		\lb with example. \hfill [3+3] (\bo{\texttt{81 Bh, 79 Bh, 78 Bh}}) [7] (73 Shr)
		\lb with flowchart and example. \hfill [6] (\texttt{78 Ka}) [10] (\bo{74 Ch})
		
		\item Draw a flowchart of floating point subtraction. \hfill [4] (70 Asa)
	\end{enumerate}
	\subsection{Subtraction Algorithm}
	\subsection{Multiplication Algorithm}
		\begin{enumerate}[noitemsep, topsep=0pt]
			\item Draw a flowchart for Booth's multiplication algorithm for signed multiplication. \hfill [4] (\texttt{78 Ka}) [5] (\bo{\texttt{81 Bh}})
			
			\item Explain booth's algorithm. \hfill [3] (\bo{70 Ch}) [4] (\bo{\texttt{80 Bh}, 68 Ch, 67 Asa}, 72 Ka) [5] (\bo{76 Ch})
			\lb with example and give hardware requirement diagram. \hfill [10] (75 Ash)
			\lb Explain with hardware algorithm with diagram. \hfill [5] (\bo{72 Ch})
			\lb Write the algorithm. \hfill [5] (76 Ash, 71 Shr)
			
			\item Design a booth multiplication algorithm hardware. \hfill [4] (\bo{71 Ch})
			
			\item Multiply using Booth's multiplication algorithm.
				\lb -6 x 7 \hfill [5] (\bo{\texttt{81 Bh}}, \texttt{80 Ba})
				\lb -7 x 3 \hfill [6] (\bo{74 Ch})
				\lb -6 x 12 \hfill [6] (72 Ka)
				\lb 10 x (-7) \hfill [6] (\texttt{81 Ba})
				\lb 10 x (-5) \hfill [5] (\bo{76 Ch})
				\lb 5 x -6 \hfill [4] (\bo{72 Ch})
				\lb (9) x (-3) \hfill [5] (71 Shr)
				\lb 23 x -21 \hfill [4] (\bo{68 Ch})
				\lb 9 x 4 \hfill [6] (\bo{\texttt{80 Bh}})
				\lb 8 x 4 \hfill [5] (76 Ash)
				\lb 8 x 9 \hfill [3] (\bo{70 Ch})
				\lb 6 x 7 \hfill [4] (\bo{67 Asa})
				\lb -7 x -10 \hfill [4] (\texttt{78 Ka})
				\lb -6 x -11 \hfill [6] (\bo{75 Ch})
				\lb -5 x -9 \hfill [5] (\bo{72 Ch})
		\end{enumerate}
	\subsection{Division Algorithm}
		\begin{enumerate}[noitemsep, topsep=0pt]
			\item How division operation can be performed? Explain with its hardware implementation. 
			\enter\hfill [10] (70 Asa)		
		
			\item Draw the flowchart for Restoring Division. \hfill [4] (\texttt{81 Ba}, 72 Ka)
			
			\item Draw the flowchart for Non-restoring Division. \hfill [4] (\bo{\texttt{79 Bh}})
			\lb Explain signed binary division algorithm. \hfill [4] (73 Shr)
			
			\item Explain non-restoring division algorithm. \hfill [3] (\bo{75 Ch}) [5] (\bo{\texttt{78 Bh}})
			\lb with flowchart. \hfill [5] (\texttt{80 Ba})
			\lb with flowchart and example. \hfill [8] (\bo{70 Ch})
			
			\item Draw the flowchart for division of floating point numbers. \hfill [4] (\bo{72 Ch, 71 Ch})
			
			\item Explain floating point division algorithm. \hfill [6] (\bo{68 Ch})
			
			\item Compare restoring division algorithm with non restoring algorithm. \hfill [4] (71 Shr) [6] (\bo{\texttt{80 Bh}}, 76 Ash, 75 Ash)
			\lb with example. \hfill [6] (\bo{76 Ch}) [8] (\bo{68 Ba})
			
			\item Divide using restoring division. 
				\lb $\dfrac{11}{5}$ \hfill [6] (\texttt{81 Ba})
				\lb 13/5 \hfill [6] (\bo{\texttt{79 Bh}})
				\lb 10/3 \hfill [7] (\bo{75 Ch})

			\item Divide using non-restoring algorithm.
				\lb 12/5 \hfill [5] (\texttt{80 Ba})
				\lb 10/5 \hfill [5] (\bo{\texttt{78 Bh}})
				\lb 15/4 \hfill [4] (73 Shr)
		\end{enumerate}
	\subsection{Logical Operation}

	\pagebreak
	
\section{Memory System}
	\begin{center}(5 Hours/8 Marks)\end{center}
	\subsection{Microcomputer Memory}
	\subsection{Characteristics of memory systems}
		\begin{enumerate}
			\item Explain in briefly the characteristics of a memory system. \hfill [3] (\texttt{81 Ba})
			\lb Write characteristics of memory system. \hfill [3] (\bo{73 Ch}) [4] (\bo{75 Ch}, 75 Ash) [8] (\bo{68 Ba})
		\end{enumerate}

	\subsection{The Memory Hierarchy}
		\begin{enumerate}
			\item What is memory hierarchy and why it is formed in computer system? \hfill [2] (71 Shr)
			
			\item Draw the memory hierarchy. \hfill [2] (\bo{72 Ch}, 72 Ka)
		\end{enumerate}

	\subsection{Internal and External memory}
	\subsection{Cache memory principles}
		\begin{enumerate}
			\item Describe the cache memory principles. \hfill [3] (\bo{\texttt{81 Bh}})

			\item Describe cache operation briefly. \hfill [2] (76 Ash)

			\item Describe cache organization. \hfill [4] (\bo{71 Ch})
		\end{enumerate}

	\subsection{Elements of Cache design}
		\begin{enumerate}
			\item Explain the various types of elements of cache design. \hfill [4] (75 Ash)
		\end{enumerate}

	\subsubsection{Cache size}
	\subsubsection{Mapping function}
		\begin{enumerate}
			\item Define cache mapping techniques. \hfill [2] (\bo{76 Ch, 68 Ba})

			\item Why cache management is necessary in mapping process? \hfill [2] (\bo{67 Asa})

			\item Explain various mapping methods used in cashe memory organization and compare each of them with example. \hfill [6] (\bo{71 Ch}) [10] (78 Ka)
			\lb only explain. \hfill [6] (75 Ash)

			\item Differentiate between direct mapping and set associative mapping. \hfill [5] (\bo{\texttt{81 Bh}}) [7] (\texttt{81 Ba})

			\item Differentiate between direct and associative mapping address structure. \hfill [6] (\bo{67 Asa})

			\item Differentiate between associative and set associative mapping with example. \hfill [5] (\bo{70 Ch})

			\item What are the major differences between different cache mapping techniques? \hfill [2] (70 Asa)
			\lb DIfferentiate among direct, associative and set associative mapping. \hfill [8] (\bo{68 Ba})

			\item Explain direct cache mapping technique with example. \hfill [7] (\bo{\texttt{80 Bh}})
			\lb with diagram. \hfill [4] (\bo{76 Ch})
			\lb with organization diagram and example. \hfill [6] (71 Shr)
			\lb with merits and demerits. \hfill [6] (\bo{72 Ch})

			\item Suppose main memory has 64 blocks and cache memory has 8 blocks when 10 blocks of main memory are used, show how mapping is performed in direct mapping technique. \hfill [6] (\bo{75 Ch})
			\lb main memory is 32 blocks, rest is same. \hfill [6] (70 Asa)
			\lb main memory is 32 blocks, cache has 8 blocks when 12 blocks are used. \hfill [6] (\bo{74 Ch})

			\item What is set associative mapping? \hfill [2] (\texttt{81 Ba}) [3] (\bo{\texttt{79 Bh}})

			\item Explain how set associative mapping technique combines the feature of direct and associative mapping technique. \hfill [3] (\texttt{81 Ba}) [5] (73 Shr)

			\item Explain about associative mapping technique. \hfill [6] (76 Ash)
			\lb with example. \hfill [6] (72 Ka)
		\end{enumerate}

	\subsubsection{Replacement algorithm}
		\begin{enumerate}
			\item Explain different replacement algorithm technique used in cache memory. \hfill [3] (\texttt{81 Ba}) [5] (\bo{\texttt{79 Bh}})

			\item Explain LRU replacement algorithm in case of hit and miss with suitable example. \hfill [8] (\bo{\texttt{78 Bh}})

			\item Why replacement algorithm is necessary in associative mapping? \hfill [4] (\bo{76 Ch})
			\lb Why replacement algorithm is used when designing cache? Explain with example.
			\enter\hfill [8] (\bo{67 Asa})
		\end{enumerate}

	\subsubsection{Write policy}
		\begin{enumerate}
			\item Explain different write policy techniques in cache memory. \hfill [3] (\bo{\texttt{80 Bh}}, 73 Shr)
		\end{enumerate}

	\subsubsection{Number of caches}

\pagebreak
\section{Input-Output organization}
	\begin{center}(6 Hours/10 Marks)\end{center}
	\subsection{Peripheral devices}
	\subsection{I/O modules}
		\begin{enumerate}
			\item What are the functions of IO Modules? \hfill [3] (71 Shr)
		\end{enumerate}

	\subsection{Input-Output interface}
		\begin{enumerate}
			\item Elaborate the roles of IO interface in a computer system. \hfill [4] (\bo{\texttt{79 Bh}, 71 Ch})

			\item Explain three reasons behind the requirement of IO interfaces. \hfill [3] (\bo{75 Ch})

			\item Explain IO interface. \hfill [2] (\bo{74 Ch})
			\lb with example. \hfill [6] (\bo{68 Ba})

			\item What are the four types of IO commands that an interface receive during the communication link between the processor and peripherals? \hfill [4] (\bo{67 Asa})

			\item Explain the IO bus and interface modules. \hfill [4] (\bo{67 Asa})
		\end{enumerate}

	\subsection{Modes of transfer}
		\begin{enumerate}
			\item Explain three IO techniques for input of a block of data. \hfill [6] (\bo{\texttt{80 Bh}})

			\item Differentiate between isolated and memory mapped IO. \hfill [4] (\bo{\texttt{78 Bh}, 72 Ch})

			\item Why memory address spaces are reduced memory mapped IO? \hfill [2] (\bo{75 Ch})
		\end{enumerate}

	\subsubsection{Programmed I/O}
		\begin{enumerate}
			\item Explain how data transfer is performed with programmed IO technique with necessary diagram.
			\enter\hfill [6] (\bo{\texttt{79 Bh}})
		\end{enumerate}

	\subsubsection{Interrupt-driven I/O}
		\begin{enumerate}
			\item Differentiate between programmed I/O and iterrupt driven I/O. \hfill [5] (\bo{\texttt{81 Bh}})

			\item How does a computer know which device issued the interrupt; if multiple devices, how does the selection take place? \hfill [5] (73 Shr)
			
			\item What are the different types of priority interrupt? \hfill [4] (72 Ka)
			
			\item What is the purpose of priority interrupt; explain priority interrupt types with key characteristics. 
			\enter\hfill [7] (71 Shr)
		\end{enumerate}

	\subsubsection{Direct Memory access}
		\begin{enumerate}
			\item With the help of a neat diagram, explain how DMA technique is used to transfer data in a computer system. \hfill [6] (\bo{\texttt{78 Bh}}) [7] (\texttt{81 Ba})
			\lb Explain DMA controller with suitable block diagram.
			\enter\hfill [5] (\bo{75 Ch}, \texttt{81 Ba}) [6] (\bo{72 Ch}, 78 Ka) [8] (76 Ash)

			\item How does DMA have request over the CPU when both request a memory transfer? \hfill [2] (76 Ash)

			\item How DMA technique is different from programmed IO? \hfill [4] (78 Ka)

			\item Compare among program IO, interrupt drive IO and DMA.
			\enter\hfill [8] (\bo{76 Ch, 74 Ch, 68 Ba}) [10] (70 Asa)

			\item Mention three possible configurations of DMA and compare them. \hfill [8] (\bo{67 Asa})
				
			\item Describe the drawbacks of programmed IO and interrupt driven IO and explain how DMA overcomes their drawbacks. \hfill [6] (\bo{71 Ch})
			\lb How does DMA overcome the problems of programmed IO and interrupt driven IO techniques?
			\enter\hfill [5] (\bo{70 Ch})
		\end{enumerate}

	\subsection{I/O Processors}
		\begin{enumerate}
			\item Explain the CPU and IOP communication channel using diagram. \hfill [5] (\bo{\texttt{81 Bh}})

			\item Why IOP is used in IO organization? \hfill [3] (\texttt{81 Ba}, 75 Ash) [5] (\bo{70 Ch}, 73 Shr)

			\item Show the role of IO processor to assist the operation of the CPU. \hfill [4] (\bo{\texttt{80 Bh}})
		\end{enumerate}

	\subsection{Data Communication Processor}
		\begin{enumerate}
			\item Why data communication processor is required in IO ogranization? \hfill [2] (\bo{76 Ch})

			\item Explain the CPU-IOP communication with diagram. \hfill [6] (72 Ka) [7] (75 Ash)

			\item Explain CPU-IOP Communication with diagram. \hfill [5] (\texttt{81 Ba})
		\end{enumerate}


\pagebreak
\section{Multiprocessors}
	\begin{center}(2 Hours/4 Marks)\end{center}
	\subsection{Characteristics of multiprocessors}
		\begin{enumerate}
			\item List out the characteristics of a multiprocessor. \hfill [4] (\bo{\texttt{80 Bh}, 70 Ch})

			\item Describe how the multiprocessor systems increase the performance level and reliability.
			\enter\hfill [4] (73 Shr)

			\item Explain about multiprocessor and multiprocessing in brief. \hfill [4] (72 Ka)

			\item How can multiprocessor be classified according to their memory organization? Explain.
			\enter\hfill [4] (\bo{71 Ch})
		\end{enumerate}

	\subsection{Interconnection Structures}
		\begin{enumerate}
			\item Discuss about loosely-coupled and tightly-cuopled architecture. \hfill [4] (\texttt{81 Ba})
			\lb Difference between them. \hfill [4] (78 Ka)

			\item Discuss about tightly-coupled multiprocessor with block diagram. \hfill [4] (76 Ash)

			\item Explain the crossbar switch interconnection structure with block diagram. \hfill [4] (\texttt{81 Ba})

			\item Differentiate Crossbar switch and Multistage switching network. \hfill [4] (71 Shr)

			\item Explain hypercube interconnection network with example. \hfill [4] (\bo{76 Ch})

			\item Compare and contrast the interconnection structures used in multiprocessing environment.
			\enter\hfill [4] (\bo{\texttt{79 Bh, 78 Bh}}) 
		\end{enumerate}

	\subsection{Interprocessor Communication and Synchronization}
		\begin{enumerate}
			\item Briefly discuss on inter-process communication and synchronization. \hfill [5] (\bo{\texttt{81 Bh}})

			\item Explain inter-process synchronization with example. \hfill [4] (\bo{74 Ch}, 70 Asa)
			\lb with suitable mechanism. \hfill [4] (\bo{72 Ch})

			\item Explain various configurations of OS in multiprocessor system. \hfill [4] (\bo{74 Ch})
		\end{enumerate}

\end{document}
