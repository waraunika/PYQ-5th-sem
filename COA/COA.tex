\documentclass[12pt]{article}
\usepackage{hyperref}
\usepackage{amsmath, amssymb, amsfonts}
\usepackage[margin=1.5cm]{geometry}
\usepackage{xcolor}
\usepackage{graphicx}
\usepackage{xparse}
\usepackage{enumitem}
\usepackage{inconsolata}
\parindent 0px
\newcommand{\lb}{\\$\left|\rightarrow\right.$}
\newcommand{\enter}{\\\textcolor{white}{1}}

\ExplSyntaxOn
\NewDocumentCommand{\bo}{m}
 {
   \bold_commas:n { #1 }
 }

\cs_new:Npn \bold_commas:n #1
 {
   \seq_set_split:Nnn \l_tmpa_seq { , } { #1 }
   \seq_map_indexed_function:NN \l_tmpa_seq \__bold_commas_aux:nn
 }

\cs_new:Npn \__bold_commas_aux:nn #1 #2
 {
   \textbf{#2}
   \int_compare:nNnTF { #1 } < { \seq_count:N \l_tmpa_seq }
     { , }
     { }
 }

\ExplSyntaxOff

\title{Computer Organization and Architecture}
\author{Me lol}
\date{\today}

\begin{document}
\maketitle
\pagebreak
\tableofcontents
\pagebreak

\section{Introduction}
	\begin{center}(3 Hours/6 Marks)\end{center}
	\subsection{Computer organization and architecture}
	\begin{enumerate}[noitemsep, topsep = 0pt]
	\item Define computer architecture.\hfill[2] (\bo{75 Ch}) [1.5] (81 Bh) [1] (\bo{72 Ch})
	\item Define computer organization.\hfill[1.5] (\bo{\texttt{81 Bh}}) [1] (\bo{72 Ch})
	\item Differentiate between computer organization and architecture.\hfill[2] (\bo{71 Ch}, 78 Ka) [3] (72 Ka)
	\item Explain the design goals and performance metrics for a computer system regarding its organization and architecture.\hspace{9.9cm}[5] (76 Ash)
	\end{enumerate}
	\subsection{Structure and function}
	\begin{enumerate}[noitemsep, topsep = 0pt]
	\item Define structure and function of a computer system.\hfill[4] (80 Ba)
	\lb Explain about the structural and functional viewpoint of a computer.\hfill[4] (\bo{\texttt{79 Bh}})
	\item Explain the functional view and four types of operations used in computer. \hfill[6] (68 Ch)
	\end{enumerate}
	\subsection{Designing for performance}
	\begin{enumerate}[noitemsep, topsep = 0pt]
	\item What are the driving factors behind the need to design for performance?\hfill[4] (71 Shr)
	\item How can we maintain a performance balance between processor and memory?\hfill[2] (\bo{72 Ch})
	\lb What is performance balance and why it is required?\hfill[3] (70 Asa)
	\end{enumerate}
	\subsection{Interconnection structures}
	\begin{enumerate}[noitemsep, topsep = 0pt]
	\item Explain the Interconnection structures of computer.\hfill[6] (75 Ash)\
	\lb Definition\hfill[2] (73 Shr)
	\lb Explain different types.\hfill[4] (73 Shr)
	\item Explain the interconnection of CPU with Memory and I/O devices along with different operations over them.\hfill[3+3] (\bo{70 Ch})
	\end{enumerate}
	\subsection{Bus interconnection}
	\begin{enumerate}[noitemsep, topsep = 0pt]
	\item What do you understand by Bus Interconnection.\hfill[2] (71 Shr)
	\item What does the width of address bus represent in a system?\hfill[2] (\bo{75 Ch, 71 Ch})
	\item Explain different elements of bus design.\hfill[2] (\bo{\texttt{79 Bh}}) [3] (70 Asa)
	\item Discuss the limitations of using single bus system to connect different devices.
	\enter\hfill[2] (\bo{75 Ch, 72 Ch})
	\item Compare and explain the bus structure of typical computer system.\hfill[4] (78 Ka)
	\item Explain different types of bus arbitration and compare them.\hfill[6] (\bo{\texttt{78 Bh}})
	\item Why is bus hierarchy required?\hfill[2] (\bo{71 Ch})
	\item Discuss about the usage of a Multiple Hierarchical Bus Architecture over single bus system.
	\enter\hfill[6] (\bo{\texttt{80 Bh}})
	\end{enumerate}
	\subsection{PCI}
	\begin{enumerate}[noitemsep, topsep = 0pt]
	\item What is PCI?\hfill[1] (76 Ash)
	\lb Describe PCI bus configuration.\hfill[3] (81 Ba)
	\end{enumerate}

	\pagebreak
\section{Central Processing Unit}
	\begin{center}(10 Hours/18 Marks)\end{center}
	\subsection{CPU Structure and Function}
	\begin{enumerate}[noitemsep, topsep = 0pt]
		\item Explain the component of CPU. \hfill [2] (\bo{\texttt{78 Ch}})
	\item Draw the instruction cycle state diagram with example.\hfill[6] (\bo{76 Ch})
	\lb Draw instruction cycle, state diagram with interrupt and explain it.\hfill[6] (\bo{74 Ch})
	\lb Explain instruction cycle state diagram.\hfill[3] (81 Ba)
	\lb Explain instruction cycle state diagram with interrupt handling.\hfill[2] (80 Ba) [3] (\bo{\texttt{81 Bh}})
	\lb Explain the computer functions with different cycles.\hfill[3] (72 Ka)

		\item Explain the general organization of register in CPU. \hfill [6] (\bo{71 Ch})
	\end{enumerate}
	\subsection{Arithmetic and Logic Unit}
	\begin{enumerate}[noitemsep, topsep = 0pt]
		\item Design a 1-bit ALU which can perform addition, AND, OR, and X-OR operations. Explain the different types of instruction formats. \hfill [4+4] (\bo{\texttt{80 Bh}})
		
		\item Design a 2-bit ALU that can perform subtraction, AND, OR and XOR. \hfill [8] (\bo{75 Ch})
		
		\item What are the stages of ALU design? Explain with the example of 2-bit ALU performing addition, subtraction, OR and XOR. \hfill [8] (70 Asa)
	\end{enumerate}
	\subsection{Instruction Formats}
	\begin{enumerate}[noitemsep, topsep = 0pt]
		\item What do you mean by instruction format? \hfill [4] (\bo{72 Ch})
		
		\item What are the different types of instructions? \hfill [3] (\bo{71 Ch})
		
		\item Explain Instruction Format with its types? \hfill [2] (71 Shr)
		
		\item What are the most common fields in an instruction. \hfill [2] (\bo{68 Ch})
		
		\item Write down the code to evaluate in three address, two address, one address and zero address instruction format.
		\begin{enumerate}[noitemsep, topsep = 0pt, label = \alph*.]
			\item Y = (A-B/C)*[D+(E*G)] \hfill [8] (\bo{\texttt{81 Bh}, 76 Ch})
			\item N = ((P-QxR)/S)+(T/U)+VxW) \hfill [8] (81 Ba)
			\item $\text{X} = \dfrac{\text{A-B+Cx(DxE-F})}{\text{G+HxK}}$ \hfill [8] (80 Ba)
			\item X = ((A+B)/C) + (D-E) \hfill [8] (\bo{\texttt{79 Bh}})
			\item Y = (A-B/C) x (D+ExG)/F \hfill [8] (78 Ka)
			\item Y = (W+X) * (Y-Z) \hfill [8] (76 Ash)
			\item Y = (A+B)/C + D/(E*F) \hfill [8] (\bo{75 Ch})
			\item (\textit{In present sources, operation is not given. If found, please contact})\hfill [8] (75 Ash)
			\item X = (P+Q) x (R+S) \hfill [8] (\bo{74 Ch})
			\item Y = A * (B+D/C)+(G*E)/F \hfill [8] (73 Shr)
			\item X = (A-B*F)*C+D/E \hfill [8] (\bo{72 Ch})
			\item Y = (A+B)*(C+D)+G/E*F \hfill [8] (72 Ka)
			\item X = (A+B) x (C+D) \hfill [5] (\bo{71 Ch}) [6] (\bo{68 Ch}, 71 Shr)
			\item Y = A/B+(CxD) + F(H/G) \hfill [8] (\bo{70 Ch}
			\item Y = A+(B*C)+D \hfill [8] (\bo{68 Ba})
			\item Y = AB + (F/G) + CD [8] (\bo{67 Asa})
			
		\end{enumerate}
	\end{enumerate}
	\subsection{Data Transfer and Manipulation}
	\begin{enumerate}[noitemsep, topsep = 0pt]
		\item Explain data transfer instruction with example. \hfill [4] (\bo{\texttt{81 Bh}})
		
		\item Explain different types of data manipulation instructions with example. \hfill [8] (\bo{\texttt{78 Bh}})
		\lb What are the three types of data manipulation instructions used in computer? Explain. 
		\enter\hfill [8] (\bo{67 Asa})
	\end{enumerate}
	\subsection{Addressing Modes}
	\begin{enumerate}[noitemsep, topsep = 0pt]
		\item What is addressing mode? \hfill [2] (\bo{\texttt{80 Bh}, 76 Ch, 68 Ch})
		
		\item Differentiate between Immediate and direct addressing modes. \hfill [4] (\bo{\texttt{81 Bh}})
		
		\item Write down the need for addressing modes. \hfill [2] (\bo{74 Ch})
		
		\item Comparision of different types of addressing modes. \hfill [6] (\bo{76 Ch}) [8] (76 Ash) [10] (72 Ka)
		\lb with adv/disadv. \hfill [10] (78 Ka)
		\lb with algorithm as well as adv/disadv. \hfill [8] (\bo{68 Ba})
		
		\item Write down different types of addressing mode and:
		\lb Explain with adv/disadv. \hfill [8] (81 Ba, 80 Ba) [10] (\bo{70 Ch})
		\lb Explain with suitable example. \hfill [6] (\bo{\texttt{80 Bh, 79 Bh}, 74 Ch}) [8] (\bo{68 Ch}, 70 Asa)
			
		\item Following instructions are give: \hfill [10] (73 Shr)
		\begin{enumerate}[noitemsep, topsep = 0pt, label = \alph*.]
			\item LDA 2000H
			\item MVI B, 32H
			\item STAX D
			\item MOV A, B
		\end{enumerate}
		Which addressing modes are used in the above instructions? Explain briefly about them.
		
		\item Describe the operation of LD (load) instruction under various addressing modes with syntax. 
		\enter\hfill [4] (\bo{71 Ch})
	\end{enumerate}
	\subsection{RISC and CISC}
	\begin{enumerate}[noitemsep, topsep = 0pt]
		\item Comparison between RISC and CISC architecture. \hfill [6] (\bo{\texttt{78 Bh}, 72 Ch}, 75 Ash)
	\end{enumerate}
	\subsection{64 – Bit Processor}

	\pagebreak
\section{Control Unit}
	\begin{center}(6 Hours/10 Marks)\end{center} 
	\subsection{Control Memory}
	
	\subsection{Addressing sequencing}
		\begin{enumerate}
			\item Explain address sequencing with the help of a block diagram. \hfill [5] (\bo{\texttt{80 Bh}})
			
			\item What is address sequencing? \hfill [3] (\bo{71 Ch, 67 Asa})
			
			\item How does a sequencing logic work in micro-programmed control unit to execute a micro-program?
			\enter\hfill [6] (\bo{70 Ch})
			\item Explain the address sequencing capabilities required in a control memory. \hfill [5] (\bo{67 Asa})
		\end{enumerate}
	
	\subsection{Computer configuration}
	
	\subsection{Microinstruction Format}
		\begin{enumerate}[noitemsep, topsep=0pt]
			\item Explain the microinstruction format.  \hfill [3] (\bo{\texttt{80 Bh}}) [4] (\texttt{81 Ba}) [5] (72 Ka)
			\lb with example. \hfill [5] (\bo{\texttt{81 Bh}}) [6] (71 Shr)
			\lb Explain various fields in micro-instruction format with neat and clean block diagram.
			\enter\hfill [3] (\bo{68 Ch})
		\end{enumerate}
		
	\subsection{Symbolic Microinstructions}
		\begin{enumerate}[noitemsep, topsep=0pt]
			\item How address of micro instruction is generated by next address generator in control unit? Explain with suitable diagram. \hfill [8] (\bo{76 Ch})
		\end{enumerate}
	
	\subsection{Symbolic Microprogram}
		\begin{enumerate}[noitemsep, topsep=0pt]
			\item Write a microprogram for the fetch cycle and addition cycle. \hfill [5] (\bo{\texttt{81 Bh}})
			
			\item Differentiate between symbolic and bianry micro instruction. \hfill [4] (\texttt{81 Ba})
			
			\item Explain the operation of microprogram sequencer used in microprogrammed control unit. 
			\enter\hfill [5] (\bo{\texttt{79 Bh}})
			
			\item Explain with diagram the working of microprogram sequencer for control memory. \hfill [6] (78 Ka)
			
			\item Write down the symbolic microprogram for fetch routine and addition execute routine.
			\enter\hfill [4] (78 Ka)
			
			\item Describe various fields in micro-instruction format with diagram showing different fields.
			\enter\hfill [6] (76 Ash)
			
			\item Write micro program for fetch cycle. \hfill [4] (73 Shr)
		\end{enumerate}
		
	\subsection{Control Unit Operation}
		\begin{enumerate}[noitemsep, topsep=0pt]
			\item What are the types of control signals? \hfill [3] (\bo{68 Ba})
		\end{enumerate}
	
	\subsection{Design of Control Unit}
		\begin{enumerate}[noitemsep, topsep=0pt]
			\item Differentiate between hardwired and microprogrammed control unit. 
			\enter\hfill [4] (\bo{74 Ch}, 70 Asa) [5] (\bo{75 Ch, 70 Ch}, \texttt{80 Ba})
			
			\item Describe the operation of hardwired control unit with a typical diagram. \hfill [5] (\bo{\texttt{79 Bh}})
			\lb Explain the key steps of hardware implementation of control unit. \hfill [7] (\bo{68 Ba})
			
			\item Explain microprogrammed control unit with block diagram. \hfill [5] (\texttt{80 Ba})
			
			\item Explain the organization structure of a microprogram control unit and the generation of control signals using microprogram. \hfill [10] (\bo{\texttt{78 Bh}})
			
			\item Explain block diagram of micro-programmed control organization. \hfill [4] (76 Ash)
			
			\item Draw and explain block diagram of micro-programmed sequencer for control memory.
			\enter\hfill [5] (\bo{75 Ch})	
			\lb Draw the diagram of Micro-programmed sequencer for a control memory and explain it. 
			\enter\hfill [10] (75 Ash)
			\lb Explain the micro program sequencer used in microprogrammed control unit. \hfill [6] (\bo{74 Ch})
			
			\item Explain microinstruction format used in microprogramming control unit. \hfill [6] (73 Shr)
			
			\item What factors cause micro-programmed control unit to be selected over hardwired control unit.
			\enter\hfill [3] (\bo{72 Ch})
			
			\item Explain with block diagram, how address of control memory is selected in micro-programmed control unit. \hfill [7] (\bo{72 Ch})
			\lb Describe how address of control memory is selected. \hfill [7] (\bo{68 Ch})
			
			\item Explain the address sequencer with the help of a block diagram. \hfill [5] (72 Ka)
			
			\item Explain the selection of address for control memory with its block diagram. \hfill [7] (\bo{71 Ch})
			
			\item Explain the organization of a control memory. \hfill [4] (71 Shr)
			
			\item Explain with steps involved when you are designing micro-program control unit. \hfill [6] (70 Asa)
		\end{enumerate}

	\pagebreak
\section{Pipeline and Vector processing}
	\begin{center}(5 Hours/10 Marks)\end{center}
	\subsection{Pipelining}
	\subsection{Parallel Processing}
	\subsection{Arithmetic Pipeline}
	\subsection{Instruction Pipeline}
	\subsection{RISC Pipeline}
	\subsection{Vector Processing}
	\subsection{Array Processing}

	\pagebreak
\section{Computer Arithmetic}
	\begin{center}(8 Hours/14 Marks)\end{center}
	\subsection{Addition Algorithm}
	\begin{enumerate}[noitemsep, topsep=0pt]
		\item Explain the floating-point addition and subtraction process
		\lb with example. \hfill [3+3] (\bo{\texttt{81 Bh, 79 Bh, 78 Bh}}) [7] (73 Shr)
		\lb with flowchart and example. \hfill [6] (\texttt{78 Ka}) [10] (\bo{74 Ch})
		
		\item Draw a flowchart of floating point subtraction. \hfill [4] (70 Asa)
	\end{enumerate}
	\subsection{Subtraction Algorithm}
	\subsection{Multiplication Algorithm}
		\begin{enumerate}[noitemsep, topsep=0pt]
			\item Draw a flowchart for Booth's multiplication algorithm for signed multiplication. \hfill [4] (\texttt{78 Ka}) [5] (\bo{\texttt{81 Bh}})
			
			\item Explain booth's algorithm. \hfill [3] (\bo{70 Ch}) [4] (\bo{\texttt{80 Bh}, 68 Ch, 67 Asa}, 72 Ka) [5] (\bo{76 Ch})
			\lb with example and give hardware requirement diagram. \hfill [10] (75 Ash)
			\lb Explain with hardware algorithm with diagram. \hfill [5] (\bo{72 Ch})
			\lb Write the algorithm. \hfill [5] (76 Ash, 71 Shr)
			
			\item Design a booth multiplication algorithm hardware. \hfill [4] (\bo{71 Ch})
			
			\item Multiply using Booth's multiplication algorithm.
				\lb -6 x 7 \hfill [5] (\bo{\texttt{81 Bh}}, \texttt{80 Ba})
				\lb -7 x 3 \hfill [6] (\bo{74 Ch})
				\lb -6 x 12 \hfill [6] (72 Ka)
				\lb 10 x (-7) \hfill [6] (\texttt{81 Ba})
				\lb 10 x (-5) \hfill [5] (\bo{76 Ch})
				\lb 5 x -6 \hfill [4] (\bo{72 Ch})
				\lb (9) x (-3) \hfill [5] (71 Shr)
				\lb 23 x -21 \hfill [4] (\bo{68 Ch})
				\lb 9 x 4 \hfill [6] (\bo{\texttt{80 Bh}})
				\lb 8 x 4 \hfill [5] (76 Ash)
				\lb 8 x 9 \hfill [3] (\bo{70 Ch})
				\lb 6 x 7 \hfill [4] (\bo{67 Asa})
				\lb -7 x -10 \hfill [4] (\texttt{78 Ka})
				\lb -6 x -11 \hfill [6] (\bo{75 Ch})
				\lb -5 x -9 \hfill [5] (\bo{72 Ch})
		\end{enumerate}
	\subsection{Division Algorithm}
		\begin{enumerate}[noitemsep, topsep=0pt]
			\item How division operation can be performed? Explain with its hardware implementation. 
			\enter\hfill [10] (70 Asa)		
		
			\item Draw the flowchart for Restoring Division. \hfill [4] (\texttt{81 Ba}, 72 Ka)
			
			\item Draw the flowchart for Non-restoring Division. \hfill [4] (\bo{\texttt{79 Bh}})
			\lb Explain signed binary division algorithm. \hfill [4] (73 Shr)
			
			\item Explain non-restoring division algorithm. \hfill [3] (\bo{75 Ch}) [5] (\bo{\texttt{78 Bh}})
			\lb with flowchart. \hfill [5] (\texttt{80 Ba})
			\lb with flowchart and example. \hfill [8] (\bo{70 Ch})
			
			\item Draw the flowchart for division of floating point numbers. \hfill [4] (\bo{72 Ch, 71 Ch})
			
			\item Explain floating point division algorithm. \hfill [6] (\bo{68 Ch})
			
			\item Compare restoring division algorithm with non restoring algorithm. \hfill [4] (71 Shr) [6] (\bo{\texttt{80 Bh}}, 76 Ash, 75 Ash)
			\lb with example. \hfill [6] (\bo{76 Ch}) [8] (\bo{68 Ba})
			
			\item Divide using restoring division. 
				\lb $\dfrac{11}{5}$ \hfill [6] (\texttt{81 Ba})
				\lb 13/5 \hfill [6] (\bo{\texttt{79 Bh}})
				\lb 10/3 \hfill [7] (\bo{75 Ch})

			\item Divide using non-restoring algorithm.
				\lb 12/5 \hfill [5] (\texttt{80 Ba})
				\lb 10/5 \hfill [5] (\bo{\texttt{78 Bh}})
				\lb 15/4 \hfill [4] (73 Shr)
		\end{enumerate}
	\subsection{Logical Operation}

	\pagebreak
\section{Memory System}
\begin{center}(5 Hours/8 Marks)\end{center}
\subsection{Microcomputer Memory}
\subsection{Characteristics of memory systems}
\subsection{The Memory Hierarchy}
\subsection{Internal and External memory}
\subsection{Cache memory principles}
\subsection{Elements of Cache design}
\subsubsection{Cache size}
\subsubsection{Mapping function}
\subsubsection{Replacement algorithm}
\subsubsection{Write policy}
\subsubsection{Number of caches}

\pagebreak
\section{Input-Output organization}
\begin{center}(6 Hours/10 Marks)\end{center}
\subsection{Peripheral devices}
\subsection{I/O modules}
\subsection{Input-Output interface}
\subsection{Modes of transfer}
\subsubsection{Programmed I/O}
\subsubsection{Interrupt-driven I/O}
\subsubsection{Direct Memory access}
\subsection{I/O Processors}
\subsection{Data Communication Processor}

\pagebreak
\section{Multiprocessors}
\begin{center}(2 Hours/4 Marks)\end{center}
\subsection{Characteristics of multiprocessors}
\subsection{Interconnection Structures}
\subsection{Interprocessor Communication and Synchronization}
\end{document}
