\documentclass[12pt]{article}
\usepackage{hyperref}
\usepackage{amsmath, amssymb, amsfonts}
\usepackage[margin=1.5cm]{geometry}
\usepackage{xcolor}
\usepackage{graphicx}
\usepackage{xparse}
\usepackage{enumitem, inconsolata}
\parindent 0px
\newcommand{\lb}{\\$\left|\rightarrow\right.$}
\newcommand{\enter}{\\\textcolor{white}{1}}

\ExplSyntaxOn
\NewDocumentCommand{\bo}{m}
 {
   \bold_commas:n { #1 }
 }

\cs_new:Npn \bold_commas:n #1
 {
   \seq_set_split:Nnn \l_tmpa_seq { , } { #1 }
   \seq_map_indexed_function:NN \l_tmpa_seq \__bold_commas_aux:nn
 }

\cs_new:Npn \__bold_commas_aux:nn #1 #2
 {
   \textbf{#2}
   \int_compare:nNnTF { #1 } < { \seq_count:N \l_tmpa_seq }
     { , }
     { }
 }

\ExplSyntaxOff

\title{Computer Network}
\author{Me lol\\Bhushan Nepal}
\date{\today}

\begin{document}
\maketitle
\vspace{13cm}
\begin{large}\textbf{Notes}\end{large}
\begin{itemize}
\item PYQs of BEX's CT657, BEI's CT613 and BCT's CT702 are combined.
\item BEX's and BCT's are kept with normal font.\item BEI's are kept with \texttt{this styling to differentiate}.
\end{itemize}
\pagebreak
\tableofcontents
\pagebreak

\section{Introduction to Computer Network}
	\begin{center}(5 Hours/8 Marks)\end{center}
	\subsection{Computer Network \& Uses}
		\begin{enumerate}[noitemsep, topsep = 0pt]
			\item What are the applications of computer networks?\hfill[2] (76 Ba)
			\item What is computer network?\hfill[1] (76 Ba) [2] (71 Shr)
			\item What are the advantages of computer network?\hfill[3] (72 Ma)
			\item Explain five instances of how networks are a part of your life today. \hfill [5] (\bo{72 Ch})
		\end{enumerate}

	\subsection{Networking model: client/server, P2P, active network}
		\begin{enumerate}[noitemsep, topsep = 0pt]
			\item Discuss Client-Server model and Peer-to-Peer model.\hfill[4] (\texttt{81 Ba})
			\item How does client-server model work?\hfill[3] (73 Ma)
			\item What are the features of Client/Server Architecture?\hfill[4] (70 Ch, 76 Ash)
			\item Draw the architecture for Client/Server network model.\hfill[2] (\bo{75 Ch})
			\item Explain briefly the architecture for peer-to-peer network model with example.\hfill[3] (\bo{77 Ch})
			\lb How does P2P works? Explain.\hfill[6] (\bo{\texttt{80 Bh}})
			\lb Explain in details about P2P network model with supportive examples.\hfill[6] (\bo{75 Ch})
			\item Distinguish between Client-Server network and Peer-Peer network.\hfill[3] (\bo{74 Ch}) [5] (73 Ma)
		\end{enumerate}

	\subsection{Network Software, Protocols and Standards}
		\begin{enumerate}[noitemsep, topsep = 0pt]
			\item Define protocol.\hfill[1] (\bo{76 Ch, 72 Ash}) [4] (bo{79 Ch}) 

			\item Explain about connection oriented and connection less service.\hfill[3] (\bo{71 Bh})
			\item Mention service primitives for implementing connection oriented service.\hfill[2] (75 Ash)
			\item What do you mean by network architecture?\hfill[2] (\bo{71 Ch})
			\item Why do you need layering?\hfill[2] (\bo{\texttt{80 Bh}})
			\item Why do we need layered protocol architecture?\hfill[2] (\bo{72 Ash})
			\item What are the reasons for using layered network architecture?\hfill[2] (\bo{76 Ch}) [3] (\bo{73 Ch}, 75 Ba)
			\item Why layering is important?\hfill[2] (\bo{\texttt{79 Bh}}, 75 Ash) [4] (\bo{73 Bh})
			\item What are the layer design issues?\hfill[3] (\bo{71 Bh})
			\item Explain design issues for layers in detail.\hfill[4] (75 Ash)
			\item Explain about the design issues of Computer Network software.\hfill[5] (\bo{74 Bh})
		\end{enumerate}

	\subsection{OSI model and TCP/IP model}
		\begin{enumerate}[noitemsep, topsep = 0pt]
			\item Explain Open System Interconnection (OSI) model.\hfill[5] (\bo{74 Ch}, 71 Ma)
			\lb Explain OSI reference model with suitable diagram. \hfill [6] (\texttt{80 Ba})
			\lb Explain the different layers of OSI reference Model with appropriate figure.\hfill[5] (\bo{76 Bh})
			\item What is the significance of OSI layer?\hfill[2] (74 Ash)
			\lb What are the reasons for using layered protocol?\hfill[3] (75 Bh)
			\item Explain in which level of OSI layer following tasks are done.\hfill[3] (\bo{72 Ch})\\
			(i) Error detection and correction \hspace{20mm} (ii) Encryption and Decryption of data\\
			(iii) Logical identification of computer \hspace{11mm} (iv) Point-to-point connection of socket\\
			(v) Dialogue control \hspace{45mm}(vi) Physical identification of computer
			\item Explain each layer of TCP/IP protocols architecture in detail.\hfill[5] (\bo{72 Ash})

			\item List out the functions of physical lapyer in TCP/IP reference model. \hfill [2] (72 Ka)

			\item Distinguish between physical channel and physical layer.\hfill[3] (\bo{74 Bh})
		\end{enumerate}

		\subsection{Comparison of OSI and TCP/IP model}
			\begin{enumerate}[noitemsep, topsep = 0pt]
				\item Differentiate between TCP/IP and OSI Model.
				\enter\hfill[3] \begin{footnotesize}(\bo{76 Bh, 71 Ch}, 71 Ma)\end{footnotesize} [4] \begin{footnotesize}(81 Ba)\end{footnotesize} [5] \begin{footnotesize}(\bo{77 Ch, 76 Ch, 72 Ch}, 76 Ba, 75 Ba, 73 Shr, 72 Ma)\end{footnotesize} [6] \begin{footnotesize}(71 Shr)\end{footnotesize} 
				\lb Explain OSI model and compare OSI with TCP/IP reference model.\hfill[6] (\bo{\texttt{79 Bh}})
				\item How do you differentiate OSI's network and transport layer with TCP/IP's Network and transport layer?\hfill[4] (\bo{79 Ch})
			\end{enumerate}

		\subsection{Data Encapsulation}
		\begin{enumerate}[noitemsep, topsep = 0pt]
			\item What is data encapsulation? \hfill [2] (\texttt{80 Ba})
			
			\item How the process of data encapsulation occurs in transmission mode described by seven layers of OSI model.\hfill[5] (75 Bh)

			\item What are headers and trailers and how do they get added and removed?
		\enter\hfill[4] (76 Ash, 70 Ch) [5] (\bo{73 Ch}) 
		\end{enumerate}

		\subsection{Example network: The Internet, X.25, Frame Relay, Ethernet, VoIP, NGN and MPLS, xDSL}
		\begin{enumerate}[noitemsep, topsep = 0pt]
		\item Explain X.25 Network with its key feature.\hfill[3] (\bo{71 Ch})
		\item Define Frame Relay in detail.\hfill[3] (73 Shr)
		\item What is Internet work?\hfill[2] (\bo{71 Bh})
		\item Explain the function of following of following devices in brief:\hfill[6] (\bo{73 Bh})\\
		(i) Hub (ii) Bridge (iii) Router
		\item You are assigned to design a network infrastructure for a 3-start hotel. Recommend a network solution with hardwares and softwares in current trend that can be used in the hotel. Make necessary assumptions and justify your recommendation with logical arguments where possible.\hfill[8] (72 Ka)
		\end{enumerate}

	\pagebreak

\section{Physical Layer}
	\begin{center}(5 Hours/8 Marks)\end{center}
	\subsection{Network monitoring: delay, latency, throughput}
		\begin{enumerate}[noitemsep, topsep=0pt]	
			\item Define:
			\lb Delay \hfill [1] (\bo{\texttt{80 Bh}}, \texttt{80 Ba})
			\lb Bandwidth \hfill [1] (\bo{\texttt{80 Bh}, 76 Bh})
			\lb MAC address \hfill [1] \bo{\texttt{80 Bh}}
			\lb Throughput \hfill [1] (\bo{76 Bh}, \texttt{80 Ba}, 76 Ba)
			\lb Latency \hfill [1] (76 Ba)
			
			\item What are the causes of packet delay? \hfill [2] (\bo{76 Bh})
		\end{enumerate}

	\subsection{Transmission media: Twisted pair, Coaxial, Fiber optic, Line-of-site, Satellite}
		\begin{enumerate}[noitemsep, topsep=0pt]
			\item What is transmission medium?  \hfill [1] (\bo{76 Ch, 71 Bh}) [2] (71 Shr) [3] (\bo{74 Ch})
			
			\item Write down the transmission medium used for networking. \hfill [4] (\bo{72 Ash})
			
			\item Explain different types of transmission media. \hfill [6] (72 Ka)
			\lb Explain about any three transmission media in detail. \hfill [6] (71 Shr)
			
			\item What are the factors to be considered while selecting transmission media? \hfill [2] (\bo{\texttt{79 Bh}}, \texttt{81 Ba})

			\item Explain twisted pair cable with its practical applications. \hfill [4] (\bo{77 Ch})	
			
			\item Explain optical fiber cable in detail with its advantages and disadvantages. \hfill [6] (\texttt{81 Ba})
			
			\item Explain different transmission medium with their merits and demerits. \hfill [6] (\bo{76 Ch})
			
			\item Explain the characteristics of twisted pair, coaxial and optical fiber cable. \hfill [6] (76 Ba)
			
			\item Explain about any two guided transmission media in detail. \hfill [6] (74 Ash)	
			
			\item List guided and unguided media used in computer network. \hfill [2] (73 Ma)
			
			\item Explain Ethernet cable standards. \hfill [6] (73 Ma)
			
			\item Compare different types of guided transmission media with appropriate figures. \hfill [6] (\texttt{80 Ba})
			
			\item Differentiate between wired and wireless media with their benefits and drawbacks. \hfill [3] (75 Ba)
			
			\item Compare among Twisted Pair, Coaxial cable and Fiber optic. \hfill [5] (\bo{74 Ch})
			
			\item Why, now a day all communication media like twisted pair, co-axial pair even wireless media are replaced by optical fibre? Justify your answer with necessary diagram, working principle and transmission mechanism. \hfill [7] (\bo{71 Bh})
			
			\item Explain various cabling techniques used in IEEE 802.3 standard. \hfill [4] (71 Ma)
		\end{enumerate}

	\subsection{Multiplexing, Circuit switching, Packet switching, VC Switching, Telecommunication switching system (Networking of Telephone exchanges)}
		\begin{enumerate}[noitemsep, topsep=0pt]
			\item What is switching? \hfill [1] (\bo{75 Ch}, 74 Ash) [2] (\bo{74 Bh}, 73 Shr)[3] (\bo{75 Bh})
			
			\item What are the various switching techniques? \hfill [2] (\bo{75 Ch})
			\lb Explain about various switchings with practical implementation example. \hfill [6] (\bo{70 Ch})
			
			\item What do you mean by data switching? \hfill [2] (\bo{70 Ch})
			
			\item Define multiplexing. \hfill [1] (74 Ash)
			
			\item Explain different types of multiplexing used in communication system. \hfill [4] (72 Ma)
			
			\item Compare switching with multiplexing. \hfill [2] (73 Shr)

			\item Elaborate packet switching with a proper diagram. \hfill [5] (\bo{75 Ch})	
			
			\item (Assumed) Discuss how data or packets goes through switch to switch in Frame Relay Virtual-circuit network. \hspace{12.6cm} [5] (\bo{\texttt{80 Bh}})
			
			\item Differentiate between datagram switching and virtual circuit switching approach. 
			\enter \hfill [4] (72 Ma) [6] (\bo{\texttt{79 Bh}})
			\lb with suitable diagram. \hfill [6] (\bo{74 Bh})
			
			\item Differentiate between circuit switching and packet switching. 
			\enter\hfill [3] (\bo{75 Bh}, 75 Ash) [4] (\bo{77 Ch, 76 Bh, 72 Ash})
			\lb Compare and which would you prefer and why? \hfill [2+2] (71 Ma)
			
			\item Discuss Packet and Circuit switching concepts with example. \hfill [5] (75 Ba)
		\end{enumerate}

	\subsection{ISDN: Architecture, Interface, and Signaling}
		\begin{enumerate}[noitemsep, topsep=0pt]
			\item What is ISDN? \hfill [2] (\bo{71 Ch})
			
			\item Explain about the ISDN architecture in detail with example. \hfill [6] (\bo{71 Ch})
			
			\item Why the telephone companies developed ISDN? \hfill [2] (76 Ash)
			
			\item Explain the working principle of ISDN with its interface and functional group. \hfill [6] (76 Ash)
			
			\item Explain ISDN channels with architecture. \hfill [5] (75 Ash)
			
			\item Explain the E1 Telephone hierarchy system. \hfill [4] (73 Shr)
		\end{enumerate}

	\pagebreak

\section{Data Link Layer}
	\begin{center}(5 Hours/8 Marks)\end{center}
	\subsection{Functions of Data link layer}
		\begin{enumerate}[noitemsep, topsep=0pt]
			\item How does data link apply flow control technique in network communication? Explain with example.
			\enter\hfill [6] (\bo{79 Ch})

			\item What are the services provided by data link layer? \hfill [2] (\bo{\texttt{80 Bh, 79 Bh, 71 Sh}}) [3] (\bo{77 Ch, 72 Ka}) [4] (\bo{74 Ch}) 

			\item What are the different sub-layers of data link layer? \hfill [2] (76 Ba)

			\item Explain the functions of each sub-layer. \hfill [6] (76 Ba)

			\item State the various design issues for the data link layer? \hfill [3] (75 Ash)
		\end{enumerate}
		
	\subsection{Framing}
		\begin{enumerate}[noitemsep, topsep=0pt]
			\item What do you mean by Framing? \hfill [2] (\bo{79 Ch}) 
			\lb Explain about framing in detail. \hfill [5] (\texttt{71 Sh})

			\item Compare Flag byte with byte stuff and bit stuffing in Framing. \hfill [2](\texttt{81 Ba}) [3] (\bo{73 Ch}) 

			\item Describe the various framing techniques at data link layer. \hfill [2] (\bo{75 Ch}) [5] (\bo{69 Bh}) [6] (\bo{75 Ch})

			\item What are the methodologies used in data framing? \hfill [4] (\bo{72 Ash}) [6] (70 Asa)

			\item How a complete link is established during the dialup connection? Explain. \hfill[4] (\bo{72 Ash})

			\item Explain any one method of framing with example. \hfill [6] (\bo{75 Ch})
		\end{enumerate}

	\subsection{Error Detection and Corrections}
		\begin{enumerate}[noitemsep, topsep=0pt]
			\item How can CRC be used to detect error? \hfill [6] (\bo{\texttt{80 Bh}})
			
			\item Detect the error if any using CRC, if received fram is 0101101101 and generator polynomial is 1001. \hfill [6] (\texttt{81 Ba})

			\item Explain Selective repeat and Go back N ARQ with example. \hfill [6] (\texttt{80 Ba})

			\item Explain difference between Error Correcting and Error detection process? \hfill [5] (\bo{70 Ch})

			\item Write down the importance of error detection and correction bits. \hfill [3] (\texttt{72 ma})

			\item Describe Cyclic Redundancy Check with example. \hfill [5] (70 Ma)
		\end{enumerate}

	\subsection{Flow Control}
		\begin{enumerate}[noitemsep, topsep=0pt]
			\item (Assumed) Differentiate between datagram switching and virtual circuit switching approach.
			\enter\hfill [4] (\bo{74 Ch}) [6] (\bo{\texttt{79 Bh}}, 74 Ash)

			\item What is piggybacking? \hfill [3] (75 Ash)

			\item (Assumed) What are the causes of packet delay in computer networks? \hfill [2] (74 Ash)

			\item What are the difference between error control and flow control? \hfill [3] (70 Ma)

			\item Explain different types of flow control mechanism in data link layer. \hfill [8] (\bo{70 Bh})

			\item (Assumed) Define switching and multiplexing. \hfill [4] (\bo{69 Ch})

			\item Differentiate between circuit switching and packet switching. \hfill [4] (\bo{69 Ch}) [5] (\bo{69 Bh})
		\end{enumerate}

	\subsection{Examples of Data Link Protocol, HDLC, PPP}
		\begin{enumerate}
			\item What is PPP? \hfill[2] (\texttt{73 Ch})
		\end{enumerate}
	\subsection{The Medium Access Sub-layer}
		\begin{enumerate}[noitemsep, topsep=0pt]
			\item (Assumed) Through we have MAC address, why do we use IP address to represent the host in networks? Explain your answer. \hfill [3] (\bo{72 Ch})

			\item Why do you think that the issues of media access is very important in data link layer? \hfill [3] (\bo{74 Bh})
		\end{enumerate}

	\subsection{The channel allocation problem}
		\begin{enumerate}[noitemsep, topsep=0pt]
			\item Why do you think that static channel assignment is not efficient? \hfill [2] (\bo{73 Ch}) 
			\item Explain the channel allocation problem with example. \hfill[5] (\bo{72 Ka})
		\end{enumerate}

	\subsection{Multiple Access Protocols}
		\begin{enumerate}[noitemsep, topsep=0pt]
			\item What are multiple access protocols? \hfill [2] (\bo{76 Bh, 75 Ch, 71 Ch})

			\item Explain how multiple access is acheived in IEEE 802.5. \hfill[6] (\bo{71 Ch})

			\item What are multiple access protocols? What is its significance in data link layer? Explain why token bus is also called as the token ring. \hfill [2+2+4] (\bo{76 Bh, 75 Ch, 73 Sh})

			\item What are hte functions of LLC and MAC sub-layer? \hfill [2+2] (70 Asa)
		\end{enumerate}

	\subsection{Ethernet}
		\begin{enumerate}
			\item Explain Ethernet frame with function of each field. \hfill [5] (\bo{77 Ch})
		\end{enumerate}

	\subsection{Networks: FDDI, ALOHA, VLAN, CSMA/CD, IEEE 802.3(Ethernet), 802.4(Token Bus), 802.5(Token Ring), and 802.1(Wireless LAN)}
		\begin{enumerate}[noitemsep, topsep=0pt]
			\item What is ALOHA system? \hfill [2] (\bo{79 Ch})
			\lb Explain different types of ALOHA. \hfill [4] (75 Ba, 72 Ma)
			\lb What is pure ALOHA and slotted ALOHA? Consider the delay of both at low load.Which one is less. \hfill [3+2] (\texttt{71 Bh})


			\item How does CSMA-CD works? Explain. \hfill [4] (\bo{76 Ch}) [6] (\bo{79 Ch})
			\lb with figure. \hfill [8] (76 Ash)

			\item Explain about the operation of CSMA/CD. \hfill [3] (\bo{74 Bh})

			\item How can you make it more efficient? \hfill [5] (\bo{74 Bh})

			\item Why is CSMA-CD not suitable for wireless medium? Explain. \hfill [2] (\bo{76 Ch})
			
			\item Explain Carrier sense multiple access with collision detection (CSMA/CD) is better than CSMA? \hfill[5] (\texttt{71 Ma})

			\item Explain about operation of Carrier Sense Multiple Access with Collision Detection. \hfill [6] (\bo{73 Ch})
			
			\item What is collission? How is it occured? \hfill [1+1] (\bo{76 Ch})

			\item How the possibility of collision is reduced in IEEE 802.3 and IEEE 802.11? Explain. \hfill [6] (\bo{76 Ch})

			\item Differentiate between Token Bus and Token Ring networks. \hfill [4] (75 Ba)

			\item List the features of FDDI. \hfill [4] (72 Ch)

			\item Explain fault tolerance mechanism of FDDI. \hfill [3] (73 Ch) 

			\item Explain IEEE 802.4. \hfill [3] (\texttt{71 Ma})
		\end{enumerate}
			
	\subsection{Numericals}
		\begin{enumerate}[noitemsep, topsep=0pt]
			\item A frame having size of 1000 bits is transmitted through a channel having capacity of 200 KB/Sec. Calculate the percentage of idleness of the channel assuming the round trip time of the channel to be 20ms. Is the channel efficient? What is your recommendation further?
			\enter\hfill [5+1+4] (\bo{73 Bh}) 

			\item A bit string 0110111011111111011111110 needs to be transmitted with flag 7E at the data linnk layer. What is the string actually transmitted after bit stuffing? \hfill [6] (\texttt{80 Ba}) [3] (\bo{73 Ch})

			\item A bit string 01111011111101111110 needs to be transmitted at the data link layer. What is the string actually transmitted after bit stuffing? \hfill [2] (75 Ash)

			\item A bit string 01111011111011111110 needs to be transmitted at the data link layer what is string actually transmitted at the data  link layer what is string actually transmitted after bit stuffing, if flag patterns is 01111110. \hfill [3] (\bo{70 Ch})
		\end{enumerate}

	\pagebreak

\section{Network Layer}
	\begin{center}(9 Hours/16 Marks)\end{center}
	\subsection{Internet working and devices: Repeaters, Hubs, Bridges, Switches, Router, Gateway}
		\begin{enumerate}
			\item Write short notes on: HUB, Switch, Routers. \hfill [4] (75 ba)
		\end{enumerate}
	\subsection{Addressing: Internet address, classful address}
		\begin{enumerate}[noitemsep, topsep=0pt]
			\item What are IPv4 address classes? \hfill [2] (\bo{79 Ch})

			\item What is classful and classless address? \hfill [2] (74 Ash)

			\item What is the Network address and broadcast address in IPv4 addressing? \hfill [2] (\texttt{81 Ba})

			\item What is private IP address? \hfill [2] (\bo{\texttt{80 Bh}, 75 Bh})

			\item What is the purpose of Time to live (TTL) and protocol field in header of IPv4 datagram.
			\enter\hfill [4] (\bo{76 Ch})
		\end{enumerate}

	\subsection{Subnetting}

	\subsection{Routing: techniques, static vs. dynamic routing , routing table for classful address}
		\begin{enumerate}
			\item What is routing? \hfill [1] (\bo{77 Ch}, 76 Ba) [2] (76 Ash)

			\item Why routing is essential in computer networking? \hfill [3] (75 Ash)
			
			\item What are the criteria for good routing? \hfill [2] (\bo{74 Ch}, \texttt{81 Ba})

			\item What is static and dynamic routing? \hfill [3] (76 Ba)
			
			\item Why do we use dynamic routing? \hfill [2] (\bo{75 Bh})

			\item What is the difference between routed and routing protocol? \hfill [2] (\bo{79 Ch})
		\end{enumerate}

	\subsection{Routing Protocols: RIP, OSPF, BGP, Unicast and multicast routing protocols}
		\begin{enumerate}
			\item Explain the general operation of RIP with timers. \hfill [6] (\texttt{81 Ba})

			\item Explain RIP, OSPF, BGP, IGRP and EIGRP. \hfill [6] (\bo{74 Ch})

			\item What do you mean by autonomous system? \hfill [2] (\bo{75 Ch})
			
			\item Why are different inter-AS and intra-AS protocols used in the internet? \hfill [2] (\texttt{80 Ba})

			\item How does link state unicast routing work? Explain. \hfill [8] (\bo{79 Ch})

			\item Explain the operation of Link State Routing Protocol. \hfill [5] (75 Ba)

			\item Define unicast and multicast routing. \hfill [2] (\bo{\texttt{79 Bh}})

			\item Which protocol is used in internet layer to provide feedback to hosts/routers about the problems in network environment? \hfill [1] (\bo{76 Ch})
		\end{enumerate}

	\subsection{Routing algorithms: shortest path algorithm, flooding, distance vector routing, link state routing; Protocols: ARP, RARP, IP, ICMP}
		\begin{enumerate}
			\item Describe flooding technique with its characteristics. \hfill [4] (\bo{77 Ch})

			\item Explain algorithm with ways to minimize the duplication of packets. \hfill [4] (\bo{\texttt{80 Bh}})

			\item Write down steps for Link State Routing Protocol. \hfill [4] (\bo{\texttt{80 Bh}})

			\item Difference between link state and distance vector routing algorithms.
			\enter\hfill [3] (\bo{77 Ch, 76 Bh}) [4] (76 Ba) [6] (\texttt{81 Ba}, 76 Ash, 74 Ash)
			\lb with example. \hfill [6] (\bo{\texttt{79 Bh}})
			\lb Compare working. \hfill [5] (75 Ash)

			\item How routing loops are prevented in distance vector routing? Explain with examples. \hfill [5] (\bo{76 Bh}) [6] (\bo{75 Ch})
			\lb Explain with example how distance vector routing is used to route the packet and why count-to-infinity problem arises and how does it get solved. \hfill [6] (\bo{75 Bh})

			\item What is ARP and how does it work? \hfill [3] (\bo{76 Ch})
		\end{enumerate}

	\subsection{Numericals}
		\begin{enumerate}
			\item Company Allegro hired an IT expert. The expert was given task to perform logical design of the company with an IP block of 206.100.100.0/24. The company had 40,20,8,100 and 5 employees in its sales, admin, finance, support and HR departments respectively. Show how he was able to perform subnetting with minimum IP wastage. \hfill [6] (\bo{\texttt{80 Bh}})

			\item Design a network for a company having 5 departments with 60, 42, 30, 10 and 12 hosts. Specifiy the network address, valid host range, broadcast address and subnet mask for each department from the given address 207.17.11.0/24. \hfill [10] (\texttt{80 Ba})
			
			\item How do you subnetwork a class C network block to following subnets of 41,14,102, and 21 computers respectively? Explain with example. \hfill [8] (\bo{79 Ch})

			\item Suppose an ISP has 200, 250, 500 and 100 customers in the four different places say, A, B, C and D and need fourr point-to-point links. Provided an IP 10.0.48.0/21, you are required to perform subnetting with minimum waste of IP. Find out the subnet masks, network address, broadcast address, usable IP range and unusable IP range for each location. \hfill [10] (\bo{\texttt{79 Bh}})

			\item Perform the subnetting of IP address block 194.53.0.0/24 for six different departments having 2, 62, 120, 5, 14 and 16 hosts. List out the subnet mask, network address, broadcast address, useable host ranges and wasted IP addresses in each subnet. \hfill [8] (\bo{77 Ch})

			\item How do you assign the sub-net IP addresses to three LANs each 12, 5 and 29 computers respectively? (Assume 202.35.91.32/25) \hspace{9cm} [6] (\texttt{81 Ba})

			\item Suppose your company has leased the IP address of 222.70.94.0/24 from your ISP. Divide it far five different departments containing 50, 30, 25, 12, 10 no of hosts. Therefore are also two points to point links far Interconnection between routers. List out the network address, broadcast address, usable IP address range and subnet mask for each subnet. Also mention the unused range of IP addresses. \hfill [8] (\bo{76 Ch})

			\item IOE has six departments having 16, 32, 61, 8, 6 and 24 computers. Use 192.168.1.0/24 to distribute the network. Find the network address, broadcast address, usable IP range and subnet mask in each department. \hfill [8] (76 Ash)

			\item  company has four departments having 20, 32, 60 and 24 computers in their respective departments. Assume an IPv4 class C public network address and design IP address blocks for each department from the assumed IP network using VLSM. Include network address, broadcast address, usable IP range and subnet mask for each of the subnet. \hfill [8] (\bo{76 Bh})

			\item Suppose you are given the IP address block of 202.101.8.0/24 from your ISP. How can you divide this IP address for four different departments of your organization requiring 50, 10, 25, 100 number of hosts with minimal waste of IP address, in each department? List out the subnet mask, network address, broadcast address and usable host addresses for each subnet. \hfill [8] (76 Ba)

			\item Suppose you are a private consultant hired by the large company to setup the network for their enterprise and you are given a large number of consecutive. UP address starting at 120.89.96.0/19. Suppose that four departments A, B, C and D request 100, 500, 800 and 400 address respectively, how the subnetting can be performed so, that adress wastage will be minimum? \hfill [8] (\bo{75 Ba})

			\item Design a network for 5 departments containing 29, 14, 15, 23 and 5 computers. Take a network example IP 202.83.54.91/25. \hfill [8] (75 Ash)

			\item You are given an IP address block of 201.40.58.0/24. Perform subnetting for four departments with equal hosts. \hfill [6] (\bo{75 Bh})

			\item How can you dedicate 32, 65, 10, 21, 9 public IP address to the departments A, B, C, D and E respectively form the pool of class C IP addresses with minimum loss. Explain. \hfill [8] (\bo{74 Ch})

			\item Suppose you are a private consultant hired by a company to setup the network for their enterprise and you are given a large number of consecutive IP address starting at 120.89.96.0/19. Suppose that four departments A, B, C and D request 100, 500, 800 and 400 addresses respectively, how the subnetting can be performed so that address wastage will be minimum? \hfill [8] (74 Ash)

			\item IOM has 4 colleges. They need to be connected in same network. Allocate following numbers of IP address: 25, 68, 19 and 50 to those colleges by reducing the losses. The IP address provided to you to allocate is 202.61.77.0/24. List the range of IP addressses, their network address, broadcast address and corresponding subnet mask. \hfill [8] (\bo{74 Bh})
		\end{enumerate}

	\pagebreak
\section{Transport Layer}
	\begin{center}(5 Hours/8 Marks)\end{center}
	\subsection{The transport service: Services provided to the upper layers}
	\subsection{Transport protocols: UDP, TCP}
	\subsection{Port and Socket}
	\subsection{Connection establishment, Connection release}
	\subsection{Flow control and buffering}
	\subsection{Multiplexing and de-multiplexing}
	\subsection{Congestion control algorithm: Token Bucket and Leaky Bucket Transport Layer}
		\begin{enumerate}
			\item What are the factors affecting Congestion? \hfill [3] (75 Ba)
		\end{enumerate}

	\pagebreak

\section{Application Layer}
	\begin{center}(5 Hours/8 Marks)\end{center}
	\subsection{Web: HTTP and HTTPS}
	\subsection{File Transfer: FTP, PuTTY, WinSCP}
	\subsection{Electronic Mail: SMTP, POP3, IMAP}
		\begin{enumerate}
			\item Compare IMAP and SMTP. \hfill [3] (75 Ba)
		\end{enumerate}
	\subsection{DNS}
	\subsection{P2PApplications}
	\subsection{Socket Programming}
	\subsection{Application server concept proxy caching, Web/Mail/DNS server optimization}
	\subsection{Concept of traffic analyzer: MRTG, PRTG, SNMP, Packet tracer, Wireshark.}

	\pagebreak

\section{Introduction to IPV6}
	\begin{center}(4 Hours/8 Marks)\end{center}
	\subsection{IPv6- Advantages}
	\subsection{Packet formats}
	\subsection{Extension headers}
	\subsection{Transition from IPv4 to IPv6: Dual stack, Tunneling, Header Translation}
	\subsection{Multicasting}

	\pagebreak

\section{Network Security}
	\begin{center}(7 Hours/16 Marks)\end{center}
	\subsection{Properties of secure communication}
	\subsection{Principles of cryptography: Symmetric Key and Public Key}
	\subsection{RSA Algorithm}
	\subsection{Digital Signatures}
	\subsection{Securing e-mail (PGP)}
	\subsection{Securing TCP connections (SSL)}
	\subsection{Network layer security (IPsec, VPN)}
	\subsection{Securing wireless LANs (WEP)}
	\subsection{Firewalls: Application Gateway and Packet Filtering, and IDS}


\end{document}
